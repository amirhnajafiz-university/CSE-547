\documentclass[12pt]{article}

% packages in use
\usepackage{amsmath, amssymb, graphicx}
\usepackage{array}
\usepackage{geometry}
\usepackage{float}

% global settings
\geometry{a4paper, margin=1in}
\setlength{\parindent}{0pt}

\newcommand{\RNum}[1]{\uppercase\expandafter{\romannumeral #1\relax}}

\begin{document}

% title section
\begin{center}
    {\LARGE\textbf{CSE 547: Homework Three}} \\[1em]
    {\large Amirhossein Najafizadeh} \\[1em]
    Semester: Fall 2024 \\ 
    SBU ID: 116715544 \\
    Email: Amirhossein.Najafizadeh@stonybrook.edu \\[1em]
    \noindent\rule{\textwidth}{0.6pt}
\end{center}

% answer sheet
\section*{Question 1}
\subsection*{1). 4.32}

\section*{Question 2}
\subsection*{2). 4.33}

\section*{Question 3}
\subsection*{3). 4.47}

\section*{Question 4}
\subsection*{4). 5.14}

\section*{Question 5}
\subsection*{5). 5.16}

\section*{Question 6}
\subsection*{6). 5.37}

\section*{Question 7}
\subsection*{7).}

\section*{Question 8}
\subsection*{8).}
Let $a$, $b$, $x$, $y$, and $v$ be integers. We want to prove that:
\[
    \gcd(a,b) \leq \gcd(xa+yb, ua+vb)
\]

Let $d = \gcd(a,b)$. By definition, $d$ is the largest integer that divides both $a$ and $b$.
Since $d$ divides both $a$ and $b$, we can express $a$ and $b$ as:
\[
    a = k_1d \quad \text{and} \quad b = k_2d, \quad \text{where } k_1 \text{ and } k_2 \text{ are integers}
\]

Now, let's consider the expression $xa + yb$:
\begin{gather*}
    xa + yb &= x(k_1d) + y(k_2d) \\
            &= d(xk_1 + yk_2)
\end{gather*}

Similarly, for $ua + vb$:
\begin{gather*}
    ua + vb &= u(k_1d) + v(k_2d) \\
            &= d(uk_1 + vk_2)
\end{gather*}

Now we can see that $d$ is a common divisor of both $xa + yb$ and $ua + vb$.
By the definition of GCD, we know that:
\[
    \gcd(xa+yb, ua+vb) \geq d
\]

Since we defined $d$ as $\gcd(a,b)$ in step 1, we can rewrite the inequality as:
\[
    \gcd(xa+yb, ua+vb) \geq \gcd(a,b)
\]

\section*{Question 9}
\subsection*{B1).}

\end{document}
