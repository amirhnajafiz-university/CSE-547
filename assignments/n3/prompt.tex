\documentclass[12pt]{article}

% packages in use
\usepackage{amsmath, amssymb, graphicx}
\usepackage{array}
\usepackage{geometry}
\usepackage{float}

% global settings
\geometry{a4paper, margin=1in}
\setlength{\parindent}{0pt}

\newcommand{\RNum}[1]{\uppercase\expandafter{\romannumeral #1\relax}}

\begin{document}

% title section
\begin{center}
    {\LARGE\textbf{CSE 547: Homework Three}} \\[1em]
    {\large Amirhossein Najafizadeh} \\[1em]
    Semester: Fall 2024 \\ 
    SBU ID: 116715544 \\
    Email: Amirhossein.Najafizadeh@stonybrook.edu \\[1em]
    \noindent\rule{\textwidth}{0.6pt}
\end{center}

% answer sheet
\section*{Question 1}
\subsection*{1). 4.32}

\section*{Question 2}
\subsection*{2). 4.33}

\section*{Question 3}
\subsection*{3). 4.47}

\section*{Question 4}
\subsection*{4). 5.14}

\section*{Question 5}
\subsection*{5). 5.16}

\section*{Question 6}
\subsection*{6). 5.37}

\section*{Question 7}
\subsection*{7).}
For any non-negative integer $n$ ($n \geq 0$), prove that the following expression is an integer:
\begin{gather*}
    (\frac{1}{5})n^5 + (\frac{1}{3})n^3 + (\frac{7}{15})n
\end{gather*}

We will use \textit{mathematical induction} to prove that the expression is an integer for all non-negative integers $n$.
When $n = 0$, the expression evaluates to 0, which is an integer.
Assume the expression is an integer for some $k \geq 0$. We need to prove it's also an integer for $k + 1$.

Let $P(k)$ represent the expression for $n = k$:
\begin{gather*}
    P(k) = (1/5)k^5 + (1/3)k^3 + (7/15)k
\end{gather*}

Now, let's calculate $P(k+1) - P(k)$:
\begin{gather*}
    P(k+1) - P(k) = [(1/5){(k+1)}^5 + (1/3){(k+1)}^3 + (7/15)(k+1)] \\
    - [(1/5)k^5 + (1/3)k^3 + (7/15)k] \\
    = [(1/5)(k^5 + 5k^4 + 10k^3 + 10k^2 + 5k + 1) + (1/3)(k^3 + 3k^2 + 3k + 1) + (7/15)(k+1)] \\
    - [(1/5)k^5 + (1/3)k^3 + (7/15)k] \\
    = k^4 + 2k^3 + k^2 + 1
\end{gather*}

This difference is always an integer for any non-negative integer $k$. Since $P(k)$ is assumed to be an integer, and the difference $P(k+1) - P(k)$ is an integer, $P(k+1)$ must also be an integer.
By the principle of \textit{mathematical induction}, we have proved that the expression is an integer for all non-negative integers $n$. \\

When $n$ is negative, the expression is not guaranteed to be an integer.
The term $(7/15)n$ will always be a fraction for any non-zero integer $n$, positive or negative.
For odd negative integers, $n^3$ and $n^5$ will be negative, potentially creating fractions that don't cancel out with the $(7/15)n$ term.
For even negative integers, $n^3$ and $n^5$ will be positive, but again, the fractions may not cancel out completely.

For $n = -1$:
\begin{gather*}
    (1/5){(-1)}^5 + (1/3){(-1)}^3 + (7/15)(-1) = -1/5 - 1/3 - 7/15 = -11/15
\end{gather*}

This is clearly not an integer. Therefore, the conclusion that the expression is always an integer does not hold for negative integers.

\section*{Question 8}
\subsection*{8).}
Let $a$, $b$, $x$, $y$, and $v$ be integers. We want to prove that:
\[
    \gcd(a,b) \leq \gcd(xa+yb, ua+vb)
\]

Let $d = \gcd(a,b)$. By definition, $d$ is the largest integer that divides both $a$ and $b$.
Since $d$ divides both $a$ and $b$, we can express $a$ and $b$ as:
\[
    a = k_1d \quad \text{and} \quad b = k_2d, \quad \text{where } k_1 \text{ and } k_2 \text{ are integers}
\]

Now, let's consider the expression $xa + yb$:
\begin{gather*}
    xa + yb = x(k_1d) + y(k_2d) \\
            = d(xk_1 + yk_2)
\end{gather*}

Similarly, for $ua + vb$:
\begin{gather*}
    ua + vb = u(k_1d) + v(k_2d) \\
            = d(uk_1 + vk_2)
\end{gather*}

Now we can see that $d$ is a common divisor of both $xa + yb$ and $ua + vb$.
By the definition of GCD, we know that:
\[
    \gcd(xa+yb, ua+vb) \geq d
\]

Since we defined $d$ as $\gcd(a,b)$ in step 1, we can rewrite the inequality as:
\[
    \gcd(xa+yb, ua+vb) \geq d = \gcd(a,b)
\]

\section*{Question 9}
\subsection*{B1).}

\end{document}
