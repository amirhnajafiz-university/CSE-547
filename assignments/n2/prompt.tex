\documentclass[12pt]{article}

% packages in use
\usepackage{amsmath, amssymb, graphicx}
\usepackage{array}
\usepackage{geometry}
\usepackage{float}

% global settings
\geometry{a4paper, margin=1in}
\setlength{\parindent}{0pt}

\newcommand{\RNum}[1]{\uppercase\expandafter{\romannumeral #1\relax}}

\begin{document}

% title section
\begin{center}
    {\LARGE\textbf{CSE 547: Homework Two}} \\[1em]
    {\large Amirhossein Najafizadeh} \\[1em]
    Semester: Fall 2024 \\ 
    SBU ID: 116715544 \\
    Email: Amirhossein.Najafizadeh@stonybrook.edu \\[1em]
    \noindent\rule{\textwidth}{0.6pt}
\end{center}

% answer sheet
\section*{Question 1}
\subsection*{1). 2.22}
First we are going to prove that $Q_{n}$ is equal to $2R_{n-1}+1$. Let's assume that we want to go from $A$ to $B$ by using reverse algorithm. In this case our available moves are:
\begin{gather*}
    B \to A \\
    A \to C \\
    C \to B
\end{gather*}

Our approach is going to be like the Tower of Hanoi's solution:
\begin{gather*}
    \text{move} (n-1) : A \to C \Rightarrow R_{n-1} \\
    \text{move the largest disk} : A \to C \to B \Rightarrow 1 \\
    \text{move} (n-1) : C \to A \Rightarrow R_{n-1} \\
    \text{overall cost is} : R_{n-1} + 1 + R_{n-1} = 2R_{n-1} + 1
\end{gather*}

Now that we proved the first relation, we are going to use the same approach to prove that $R_{n}$ is equal to $Q_{n}+Q_{n-1}+1$. Assume we want to go back to $B$ from $A$:
\begin{gather*}
    \text{move} (n-1) : B \to A \Rightarrow R_{n-1} \\
    \text{move} (n-1) : A \to C \Rightarrow R_{n-1} \\
    \text{move the largest disk} : B \to A \Rightarrow 1 \\
    \text{move n-1 disks (it's like two reverse move in order)} : C \to B \to A \Rightarrow 2R_{n-2}+2 \\
    \text{overall cost is} : R_{n-1} + R_{n-1} + 1 + 2R_{n-2} + 2 \\= (2R_{n-1} + 1) + (2R_{n-2} + 1) + 1 = Q_{n} + Q_{n-1} + 1
\end{gather*}

Not to mention that in all cases when $n=0$, then we don't have any disks to move. Therefore, the equations for $n=0$ remains as $R_{0}=Q_{0}=0$.

\section*{Question 2}
\subsection*{3). 4.16}
As this is like the \textit{Line in the plane} problem, we are going to go with the same approach.
\begin{gather*}
    P(n) = P(n-1) + \text{new spaces created by adding the $n$th plane}
\end{gather*}

Now, for counting the new spaces created by adding the new plane, the nth plane can intersect each of the previous $n-1$ planes in a line.
These intersections divide nth plane into new spaces. So, it is a combination of the new and old planes. Therefore:
\begin{gather*}
    P(n) = P(n-1) + \begin{pmatrix} n \\ 2 \end{pmatrix} + n + 1 \\
    P(n) = P(n-1) + \frac{n \times (n-1)}{2} + n + 1
\end{gather*}

For solving $P(n)$, we are going to use the \textit{Line in the plane} solution:
\begin{gather*}
    P(n) = P(n-1) + \frac{n \times (n-1)}{2} + n + 1 \\
    = P(n-2) + \frac{(n-1) \times (n-2)}{2} + (n-1) + 1 + \frac{n \times (n-1)}{2} + n + 1  \\
    = P(n-3) + \frac{(n-2) \times (n-3)}{2} + (n-2) + 1 + \frac{(n-1) \times (n-2)}{2} + (n-1) + 1 + \frac{n \times (n-1)}{2} + n + 1  \\
    = 1 + \sum_{j=0}^n 1 + \sum_{j=0}^n \frac{j}{2} + \sum_{j=0}^n \begin{pmatrix} n-j \\ 2 \end{pmatrix} \\
    = 1 + n + \frac{n \times (n-1)}{2} + \frac{n \times (n-1) \times (n-2)}{2 \times 3}
\end{gather*}

To test our answer:
\begin{gather*}
    P(1) = 1 \\
    P(2) = 1 + 2 + 1 = 4 \\
    P(3) = 1 + 3 + \frac{3 \times (3-1)}{2} + \frac{3 \times (3-1) \times (3-2)}{6} = 8
\end{gather*}

\section*{Question 3}
\subsection*{4). 4.24}
First we are going to simplify our recurrence function to use the repertoire method.
\begin{gather*}
    g(1) = \alpha \text{ $(n=0)$} \\
    g(2n+j) = 3g(n) + \gamma n + \beta_{j} \text{ $(n \geq 1, j=0,1)$} \\
    \to g(2n) = 3g(n) + \gamma n + \beta_{0} \\
    \to g(2n+1) = 3g(n) + \gamma n + \beta_{1} \\
    \Rightarrow g(n) = A(n) \alpha + B(n) \gamma + C(n) \beta_{0} + D(n) \beta_{1}
\end{gather*}

Now first we are going to assign $\gamma = 0$, then we have:
\begin{gather*}
    g(1) = \alpha \\
    g(2n) = 3g(n) + \beta_{0} \\
    g(2n+1) = 3g(n) + \beta_{1} \\
    \Rightarrow g(n) = A(n) \alpha + C(n) \beta_{0} + D(n) \beta_{1}
\end{gather*}

After removing gamma, now we are going to solve this three-parameters equation by using repertoire method.
First we are finding $g(n)$ by using small values of $n$. As in the table bellow, we are going to start
from $n=1$ to $n=11$.
\begin{table}[H]
    \centering
    \begin{tabular}{c|c}
        $n$ & $g(n)$ \\
        \hline
        $1$ & $\alpha$ \\
        \hline
        $2$ & $3\alpha + \beta_{0}$ \\
        $3$ & $3\alpha + \beta_{1}$ \\
        \hline
        $4$ & $9\alpha + 4\beta_{0}$ \\
        $5$ & $9\alpha + 3\beta_{0} + \beta_{1}$ \\
        $6$ & $9\alpha + \beta_{0} + 3\beta_{1}$ \\
        $7$ & $9\alpha + 4\beta_{1}$ \\
        \hline
        $8$ & $27\alpha + 13\beta_{0}$ \\
        $9$ & $27\alpha + 12\beta_{0} + \beta_{1}$ \\
        $10$ & $27\alpha + 10\beta_{0} + 3\beta_{1}$ \\
        $11$ & $27\alpha + 9\beta_{0} + 4\beta_{1}$
    \end{tabular}
    \caption{general table for small values of $n$}
    \label{tab:sample}
\end{table}

As it looks, we are going to have:
\begin{gather*}
    A(n) = 3^m, \text{where } n = 2^m + l \text{ and } 0 \leq l < 2^m \\
    A(n) - C(n) - D(n) = 1 \\
    A(n) + C(n) = n
\end{gather*}

Therefore, we have $A(n), C(n), and D(n)$. Now for finding $B(n)$, we are going to use $g(n)=n$.
\begin{gather*}
    n = A(n) \alpha + B(n) \gamma + C(n) \beta_{0} + D(n) \beta_{1} \\
    \text{a satisfying value set is } (\alpha, \gamma, \beta_{0}, \beta_{1}) = (1, -1, 0, 1) \\
    \to n = A(n) - B(n) + D(n) \to B(n) = A(n) + D(n) - n
\end{gather*}

Now that we have every parameter written based on others, we have:
\begin{gather*}
    A(n) = 3^m, \text{where } n = 2^m + l \text{ and } 0 \leq l < 2^m \\
    C(n) = n - 3^m = 2^m - 3^m + l \\
    D(n) = A(n) - 1 - C(n) = 3^m - 1 - 2^m + 3^m - l = 2 \time 3^m - 2^m - l \\
    B(n) = A(n) + D(n) - n = 3^m + 2 \time 3^m - 2^m - l - n = 3^{m+1} - 2^{m+1} -2l
\end{gather*}


\section*{Question 4}
\subsection*{5). 4.26}
In order to see if an answer exists for this problem, first we are going to check what is our position after $k$th step. We start from number 1 and each time we move $m$ places to kill the person on that position.
To meet the condition of our problem, we need to make sure that our position is always in $(n, n+1, \ldots, 2n)$th positions. Now let's see what are the available positions in each step:
\begin{table}[h]
    \centering
    \begin{tabular}{|c|c|}
        \hline
        step & available positions \\
        \hline
        $1$ & $P(1): m \mod 2n \in (n, \ldots, 2n)$ \\
        \hline
        $2$ & $P(2): P(1) + m \mod {2n-1} \in (n, \ldots, 2n-1)$ \\
        \hline
        $3$ & $P(3): P(2) + m \mod {2n-2} \in (n, \ldots, 2n-2)$ \\
        \hline
        $\ldots$ & $\ldots$ \\
        \hline
        $k$ & $P(k): P(k-1) + m \mod {2n-k+1} \in (n, \ldots, 2n-k+1)$ \\
        \hline
    \end{tabular}
    \caption{available positions after each step}
    \label{tab:sample}
\end{table}

In general:
\begin{gather*}
    P(k) = P(k-1) + m \mod {(2n-k+1)} \\
    = P(k-2) + m \mod {(2n-k)} + m \mod {(2n-k+1)} \\
    = m \mod 2n + m \mod 2n + m \mod {(2n-1)} + \ldots
\end{gather*}

Based on the available positions, $m$ should be a number that $P(k) \text{is always in} [n+1, 2n-k]$. To meet that, it needs a common multiple with each number
in the equation. Therefore, $m$ can be any number that is in the common multiple of number $n+1, n+2, \ldots, 2n$.

\section*{Question 5}
\subsection*{6). 4.30}
First we try to convert our summation:
\begin{gather*}
    \sum_{k=0}^n k2^k \to \sum_{1 \leq j \leq k \leq n } 2^k \\
    \text{($\RNum{1}$) we know that: } \sum_{j=0}^n 1 = n \\
    \text{therefore, we can replace the innser $k$ with ($\RNum{1}$): } k = \sum_{j=0}^k 1 \\
    S_{n} = \sum_{1 \leq j \leq k \leq n} 2^k
\end{gather*}

Now to solve the problem, according to the textbook, we have:
\begin{gather*}
    \text{$(\RNum{1})$ } 2S_{UT} = \sum_{1 \leq j,k \leq n} a_{j}a_{k} - \sum_{1 \leq j=k \leq n} a_{j}a_{k} \\
    \text{$(\RNum{2})$ } \sum_{k=0}^n 2^k = 2^{n+1} - 1
\end{gather*}

As for this problem:
\begin{gather*}
    2S_{UT} = \sum{1 \leq j,k \leq n} 2^k - \sum{1 \leq j=k \leq n} 2^k \\
    \text{$(\RNum{1}), (\RNum{2})$ } 2S_{UT} = n2^{n+1} - 1 - (2^{n+1} + 1) \\
    2S_{UT} = n2^{n} \times 2 - 2^n \times 2 - 2 \\
    S_{UT} = S_{n} = n2^{n} - 2^n - 1 
\end{gather*}

\section*{Question 6}
\subsection*{7). 4.38}
To solve these summations using perturbation method, we need to take the first and last element out of the summation. Let's start from
$S_{n}$ because the two other summations are an extended version of it.
\begin{gather*}
    S_{n} = \sum_{k=0}^n {-1}^{n-k} \\
    S_{n+1} = 1 - S_{n} = {-1}^{n+1} + \sum_{k=1}^{n+1} {-1}^{n+1-k} = {-1}^{n+1} + \sum_{k=0}^n {-1}^{n-k} \\
    = {-1}^{n+1} + S_{n} \Rightarrow 2S_{n} = 1 - {-1}^{n+1} \to S_{n} = \frac{1}{2} (1 + {-1}^n) \\
    S_{n} = \begin{cases} 0 & \text{if n is odd} \\ 1 & \text{if n is even} \end{cases}
\end{gather*}

Now we are going to solve $T_{n}$ and $U_{n}$ with the same approach:
\begin{gather*}
    T_{n} = \sum_{k=0}^n {-1}^{n-k} k \\
    T_{n+1} = n + 1 - T_{n} = \sum_{k=0}^{n} {-1}^{n-k} (k+1) = \sum_{k=0}^n {-1}^{n-k} + \sum_{k=0}^n {-1}^{n-k} k \\
    = S_{n} + T_{n} \Rightarrow 2T_{n} = n + 1 - S_{n} \to T_{n} = \frac{1}{2} (n + 1 - S_{n}) \\
    T_{n} = \begin{cases} \frac{n+1}{2} & \text{if n is odd} \\ \frac{n}{2} & \text{if n is even} \end{cases}
\end{gather*}

\begin{gather*}
    U_{n} = \sum_{k=0}^n {-1}^{n-k} k^2 \\
    U_{n+1} = {(n + 1)}^2 - U_{n} = \sum_{k=0}^{n} {-1}^{n-k+1} {(k+1)}^2 = \sum_{k=0}^n {-1}^{n-k} + \sum_{k=0}^n {-1}^{n-k} {(k+1)}^2 \\
    = S_{n} + 2T_{n} + U_{n} \Rightarrow 2U_{n} = n^2 + 2n + 1 - 2T_{n} - S_{n} \\
    T_{n} = \begin{cases} n^2 + n & \text{if n is odd} \\ n^2 + n & \text{if n is even} \end{cases}
\end{gather*}

\section*{Question 7}
\subsection*{9).}
If we have $m\text{ $|$ }n$ (meaning that $n$ can be divided by $m$), then n should be divided by all prime factors of n.
\begin{gather*}
    m = {p_{1}}^{a_{1}} \times {p_{2}}^{a_{2}} \ldots \times {p_{k}}^{a_{k}} \\
    m \text{ $|$ } n \Rightarrow {p_{1}}^{a_{1}} \times {p_{2}}^{a_{2}} \ldots \times {p_{k}}^{a_{k}} \text{ $|$ } n \\
    {p_{1}}^{a_{1}} \text{ $|$ } n \wedge {p_{2}}^{a_{2}} \text{ $|$ } n \ldots \wedge {p_{k}}^{a_{k}} \text{ $|$ } n
\end{gather*}

In this problem, we can write number six as its prime factors $2 \times 3$. Therefore, we need need to prove that:
\begin{gather*}
    2 \text{ $|$ } n \times (n+1) \times (n+2) \wedge 3 \text{ $|$ } n \times (n+1) \times (n+2)
\end{gather*}

Since 2 and 3 are coprime, then we can prove each of them individually. Let's start with 2:
\begin{gather*}
    \text{if } n \equiv 0 \mod 2 \to n \times (n+1) \times (n+2) \equiv 0 \mod 2 \\
    \text{if } n \equiv 1 \mod 2 \to n+1 \equiv 2 \equiv 0 \mod 2 \to n \times (n+1) \times (n+2) \equiv 0 \mod 2
\end{gather*}

To say it in simple words, for each real number as $N$, one of the two $N$ or $N+1$ is divided by 2. As for 3:
\begin{gather*}
    \text{if } n \equiv 0 \mod 3 \to n \times (n+1) \times (n+2) \equiv 0 \mod 3 \\
    \text{if } n \equiv 1 \mod 3 \to n+2 \equiv 3 \equiv 0 \mod 3 \to n \times (n+1) \times (n+2) \equiv 0 \mod 3 \\
    \text{if } n \equiv 2 \mod 3 \to n+1 \equiv 3 \equiv 0 \mod 3 \to n \times (n+1) \times (n+2) \equiv 0 \mod 3
\end{gather*}

To say it in simple words, for each real number as $N$, one of the three $N$, $N+1$, or $N+2$ is divided by 3. Now we can say:
\begin{gather*}
    2 \text{ $|$ } n \times (n+1) \times (n+2) \wedge 3 \text{ $|$ } n \times (n+1) \times (n+2) \\
    6 \text{ $|$ } n \times (n+1) \times (n+2)
\end{gather*}

\section*{Question 8}
\subsection*{10).}
In order to prove this equation, we are going to use a \textit{mathematical induction}. In the Fibonacci sequence we have:
\begin{gather*}
    F_{n} = F_{n-1} + F_{n-2} \\
    F_{0}=0, F_{1}=1, F_{2}=1
\end{gather*}

For $n=1$ we have:
\begin{gather*}
    \begin{bmatrix}
        F_{2} & F_{1} \\
        F_{1} & F_{0}
    \end{bmatrix}
    =
    \begin{bmatrix}
        1 & 1 \\
        1 & 0
    \end{bmatrix}
    =
    \left(\begin{bmatrix}
        1 & 1 \\
        1 & 0
    \end{bmatrix}\right)^1
\end{gather*}

Now let's assume for $n-1$ we have:
\begin{gather*}
    \begin{bmatrix}
        F_{n} & F_{n-1} \\
        F_{n-1} & F_{n-2}
    \end{bmatrix}
    =
    \left(\begin{bmatrix}
    1 & 1 \\
    1 & 0
    \end{bmatrix}\right)^{n-1}
\end{gather*}

Now we are going to multiply both sides by \( \begin{bmatrix} 1 & 1 \\ 1 & 0 \end{bmatrix} \):
\begin{gather*}
    \begin{bmatrix}
        F_{n} & F_{n-1} \\
        F_{n-1} & F_{n-2}
    \end{bmatrix}
    \times
    \begin{bmatrix}
        1 & 1 \\
        1 & 1
    \end{bmatrix}
    =
    \begin{bmatrix}
        F_{n}+F_{n-1} & F_{n} \\
        F_{n-1}+F_{n-2} & F_{n-1}
    \end{bmatrix}
    =
    \begin{bmatrix}
        F_{n+1} & F_{n} \\
        F_{n} & F_{n-1}
    \end{bmatrix}
    =
    \left(\begin{bmatrix}
    1 & 1 \\
    1 & 0
    \end{bmatrix}\right)^{n}
\end{gather*}

And now that we proved the $n$th step, we can say by the power of \textit{mathematical induction}:
\begin{gather*}
    \begin{bmatrix}
        F_{n+1} & F_{n} \\
        F_{n} & F_{n-1}
    \end{bmatrix}
    =
    \left(\begin{bmatrix}
    1 & 1 \\
    1 & 0
    \end{bmatrix}\right)^{n}
\end{gather*}

\end{document}
