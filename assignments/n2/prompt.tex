\documentclass[12pt]{article}

% packages in use
\usepackage{amsmath, amssymb, graphicx}
\usepackage{array}
\usepackage{geometry}
\usepackage{float}

% global settings
\geometry{a4paper, margin=1in}
\setlength{\parindent}{0pt}

\newcommand{\RNum}[1]{\uppercase\expandafter{\romannumeral #1\relax}}

\begin{document}

% title section
\begin{center}
    {\LARGE\textbf{CSE 547: Homework Two}} \\[1em]
    {\large Amirhossein Najafizadeh} \\[1em]
    Semester: Fall 2024 \\ 
    SBU ID: 116715544 \\
    Email: Amirhossein.Najafizadeh@stonybrook.edu \\[1em]
    \noindent\rule{\textwidth}{0.6pt}
\end{center}

% answer sheet
\section*{Question 1}
\subsection*{1). 2.22}
Let's review Lagrange's identity first:
\begin{gather*}
    \sum_{1 \leq j < k \leq n} {(a_{j}b_{k} - a_{k}b_{j})}^2 = (\sum_{k=1}^n {a_{k}}^2)(\sum_{k=1}^n {b_{k}}^2) - {(\sum_{k=1}^n a_{k}b_{k})}^2
\end{gather*}

First we are going to prove Lagrange's identity, after that we are going find an identity
for the more general double sum. We are going to expand each part of the equation, then combine
everything to together.
\begin{gather*}
    \text{($\RNum{1}$) } {(\sum_{i=1}^n a_{i}b_{i})}^2 = \sum_{i=1}^n \sum_{j=1}^n a_{i}b_{i}a_{j}b_{j} = \sum_{i=1}^n {a_{i}}^2 {b_{i}}^2 + 2 \sum_{1 \leq i < j \leq n}^n (a_{i}b_{i}a_{j}b_{j}) \\
    \\
    \text{($\RNum{2}$) } (\sum_{i=1}^n {a_{i}}^2)(\sum_{i=1}^n {b_{i}}^2) = \sum_{i=1}^n {a_{i}}^2 {b_{i}}^2 + \sum_{1 \leq i < j \leq n} ({a_{i}}^2 {b_{j}}^2 + {a_{j}}^2 {b_{i}}^2) \\
    \\
    \text{($\RNum{3}$) } - \sum_{1 \leq i < j \leq n} {(a_{i}b_{j} - a_{j}b_{i})}^2 = - \sum_{1 \leq i < j \leq n} ({a_{i}}^2 {b_{j}}^2 - 2a_{i}b_{i}a_{j}b_{j} + {a_{j}}^2 {b_{i}}^2) \\
    \\
    \text{($\RNum{2}$) + ($\RNum{3}$) } \sum_{i=1}^n {a_{i}}^2 {b_{i}}^2 + \sum_{1 \leq i < j \leq n} ({a_{i}}^2 {b_{j}}^2 + {a_{j}}^2 {b_{i}}^2) - \sum_{1 \leq i < j \leq n} ({a_{i}}^2 {b_{j}}^2 - 2a_{i}b_{i}a_{j}b_{j} + {a_{j}}^2 {b_{i}}^2) \\
    = \sum_{i=1}^n {a_{i}}^2 {b_{i}}^2 + 2 \sum_{1 \leq i < j \leq n} a_{i}b_{i}a_{j}b_{j}
\end{gather*}

Since we have $\RNum{1} = \RNum{2} + \RNum{3}$; it proves Lagrange's identity.
\begin{gather*}
    {(\sum_{i=1}^n a_{i}b_{i})}^2 = (\sum_{i=1}^n {a_{i}}^2)(\sum_{i=1}^n {b_{i}}^2) - \sum_{1 \leq i < j \leq n} {(a_{i}b_{j} - a_{j}b_{i})}^2 \\
    \sum_{1 \leq i < j \leq n} {(a_{i}b_{j} - a_{j}b_{i})}^2 = (\sum_{i=1}^n {a_{i}}^2)(\sum_{i=1}^n {b_{i}}^2) - {(\sum_{i=1}^n a_{i}b_{i})}^2
\end{gather*}

Now we are going to use it to find an identity for this summation:
\begin{gather*}
    S = \sum_{1 \leq j < k \leq n} ((a_{j}b_{k} - a_{k}b_{j})(A_{j}B_{k} - A_{k}B_{j}))
\end{gather*}

According to Lagrange's identity we have for ${a},{b},{A},{B}$:
\begin{gather*}
    (\sum_{i=1}^n a_{i}b_{i})(\sum_{i=1}^n A_{i}B_{i}) = (\sum_{i=1}^n a_{i}A_{i})(\sum_{i=1}^n b_{i}B_{i}) - \sum_{1 \leq j < k \leq n} (a_{j}b_{k} - a_{k}b_{j})(A_{j}B_{k} - A_{k}B_{j}) \\
    \sum_{1 \leq j < k \leq n} (a_{j}b_{k} - a_{k}b_{j})(A_{j}B_{k} - A_{k}B_{j}) = (\sum_{i=1}^n a_{i}A_{i})(\sum_{i=1}^n b_{i}B_{i}) - (\sum_{i=1}^n a_{i}b_{i})(\sum_{i=1}^n A_{i}B_{i})
\end{gather*}

\section*{Question 2}
\subsection*{3). 4.16}
First of all, let's take a look at Euclid's theorem:
\begin{gather*}
    E_{n} = \prod_{i=1}^n p_{i} + 1 \text{ where $p_{i}$ is the $i$th prime number}
\end{gather*}

Now we are going to find a recursive definition for Euclid's numbers:
\begin{gather*}
    E_{n+1} = \prod_{i=1}^{n+1} p_{i} + 1 \\
    E_{n+1} = p_{n+1} \prod_{i=1}^n p_{i} + 1 \\
    E_{n+1} = E_{n} \times p_{n+1} + 1 \\
    \Rightarrow E_{n} = E_{n-1} \times p_{n} + 1 \\
    E_{n} = E_{1} \ldots E_{n-1} + 1
\end{gather*}

Now we are going to work on the sum of the reciprocals of the first $n$ Euclid number.
\begin{gather*}
    S_{n} = \sum_{i=0}^n \frac{1}{E_{i}} \\
    = \frac{1}{E_{1}} + \cdots + \frac{1}{E_{n}} \\
    = 1 - \frac{1}{E_{1} \ldots E_{n}} = 1 - \frac{1}{E_{n+1} - 1}
\end{gather*}

\section*{Question 3}
\subsection*{4). 4.24}
Let's express $n$ in base $p$:
\begin{gather*}
    n = d_{k}p^k + d_{k-1}p^{k-1} + \cdots + d_{0} \\
    \text{The sum of the digits is: } \nu_p(n) = d_{k} + d_{k-1} + \cdots + d_{0}
\end{gather*}

Let $e_p(n!)$ represent the sum of the digits of $n!$ in base $p$, and $\nu_p(n!)$ represent the highest power of $p$ dividing $n!$
\begin{gather*}
    e_p(n!) = \text{sum of the digits of } n! \text{ in base } p \\
    \nu_p(n!) = \sum_{i=1}^{\infty} \left\lfloor \frac{n}{p^i} \right\rfloor \\
    \nu_p(n) = \text{sum of the digits of } n \text{ in base } p
\end{gather*}

The term $\nu_p(n)$ adjusts the total count of digits by accounting for the contributions from the
factors of $p$ in $n$. Now the term $\frac{\nu_p(n)}{p-1}$ represents how many full sets of $p$ are in $n$.
Consider the contribution of a digit $d$ in position $m$ in base $p$. The contribution of this digit to $e_p(n!)$ is:

\begin{gather*}
    d \cdot (p^m + p^{m-1} + \cdot + p^0) = d \cdot \frac{p^{m+1} - 1}{p - 1}
\end{gather*}

This formula arises from the geometric series for summing powers of $p$.
The sum of the digits of $n!$ in base $p$ can be expressed as:

\begin{gather*}
    e_p(n!) = n - \frac{\nu_p(n)}{p - 1}
\end{gather*}

Where $n$ is the total number of digits and $\nu_p(n)$ counts how many times $p$ divides into $n$.

\section*{Question 4}
\subsection*{5). 4.26}
In order to see if an answer exists for this problem, first we are going to check what is our position after $k$th step. We start from number 1 and each time we move $m$ places to kill the person on that position.
To meet the condition of our problem, we need to make sure that our position is always in $(n, n+1, \ldots, 2n)$th positions. Now let's see what are the available positions in each step:
\begin{table}[h]
    \centering
    \begin{tabular}{|c|c|}
        \hline
        step & available positions \\
        \hline
        $1$ & $P(1): m \mod 2n \in (n, \ldots, 2n)$ \\
        \hline
        $2$ & $P(2): P(1) + m \mod {2n-1} \in (n, \ldots, 2n-1)$ \\
        \hline
        $3$ & $P(3): P(2) + m \mod {2n-2} \in (n, \ldots, 2n-2)$ \\
        \hline
        $\ldots$ & $\ldots$ \\
        \hline
        $k$ & $P(k): P(k-1) + m \mod {2n-k+1} \in (n, \ldots, 2n-k+1)$ \\
        \hline
    \end{tabular}
    \caption{available positions after each step}
    \label{tab:sample}
\end{table}

In general:
\begin{gather*}
    P(k) = P(k-1) + m \mod {(2n-k+1)} \\
    = P(k-2) + m \mod {(2n-k)} + m \mod {(2n-k+1)} \\
    = m \mod 2n + m \mod 2n + m \mod {(2n-1)} + \ldots
\end{gather*}

Based on the available positions, $m$ should be a number that $P(k) \text{is always in} [n+1, 2n-k]$. To meet that, it needs a common multiple with each number
in the equation. Therefore, $m$ can be any number that is in the common multiple of number $n+1, n+2, \ldots, 2n$.

\section*{Question 5}
\subsection*{6). 4.30}
First we try to convert our summation:
\begin{gather*}
    \sum_{k=0}^n k2^k \to \sum_{1 \leq j \leq k \leq n } 2^k \\
    \text{($\RNum{1}$) we know that: } \sum_{j=0}^n 1 = n \\
    \text{therefore, we can replace the innser $k$ with ($\RNum{1}$): } k = \sum_{j=0}^k 1 \\
    S_{n} = \sum_{1 \leq j \leq k \leq n} 2^k
\end{gather*}

Now to solve the problem, according to the textbook, we have:
\begin{gather*}
    \text{$(\RNum{1})$ } 2S_{UT} = \sum_{1 \leq j,k \leq n} a_{j}a_{k} - \sum_{1 \leq j=k \leq n} a_{j}a_{k} \\
    \text{$(\RNum{2})$ } \sum_{k=0}^n 2^k = 2^{n+1} - 1
\end{gather*}

As for this problem:
\begin{gather*}
    2S_{UT} = \sum{1 \leq j,k \leq n} 2^k - \sum{1 \leq j=k \leq n} 2^k \\
    \text{$(\RNum{1}), (\RNum{2})$ } 2S_{UT} = n2^{n+1} - 1 - (2^{n+1} + 1) \\
    2S_{UT} = n2^{n} \times 2 - 2^n \times 2 - 2 \\
    S_{UT} = S_{n} = n2^{n} - 2^n - 1 
\end{gather*}

\section*{Question 6}
\subsection*{7). 4.38}
First let's take a look at the equation:
\begin{gather*}
    \gcd((a^n - b^n), (a^m - b^m)) = a^{\gcd(n,m)} - b^{\gcd(n,m)} \\
    a \perp b, a > b, 0 \leq m < n
\end{gather*}

Now we are going to check some lemmas before proving the equation. First, we know that:
\begin{gather*}
    \gcd(a, b) = \gcd(a, b - k \times a)
\end{gather*}

Next, we have a polynomial identity for $a^n - b^n$:
\begin{gather*}
    (a^n - b^n) = (a - b)(a^{n-1}b^0 + \cdots + a^0b^{n-1})
\end{gather*}

So let's assume that $r = n \mod m$. As the polynomial identity displayed, we can have:
\begin{gather*}
    (a^n - b^n) = (a^m - b^m)(a^{n-m}b^0 + \cdots + a^r b^{n-m-r}) + b^{m \left\lfloor \frac{n}{m} \right\rfloor} (a^r - b^r)
\end{gather*}

Now, we are going replace this in the first equation:
\begin{gather*}
    \gcd((a^m - b^m), (a^m - b^m)(a^{n-m}b^0 + \cdots + a^r b^{n-m-r}) + b^{m \left\lfloor \frac{n}{m} \right\rfloor} (a^r - b^r)) \\
    = \gcd((a^m - b^m), b^{m \left\lfloor \frac{n}{m} \right\rfloor} (a^r - b^r))
\end{gather*}

Since we have $a \perp b$, it means that $b^{m \left\lfloor \frac{n}{m} \right\rfloor}$ has no common factors in this equation.
So we are going to remove it and have:
\begin{gather*}
    \gcd((a^n - b^n), (a^m - b^m)) = \gcd((a^m - b^m), (a^r - b^r)) \text{ where $r = n \mod m$}
\end{gather*}

If we continue this approach, we can go until we reach $\gcd(m,n)$. Since the next number after that will
be zero, we can prove the validation of our equation:
\begin{gather*}
    \gcd((a^n - b^n), (a^m - b^m)) = \gcd((a^m - b^m), (a^r - b^r)) \text{ where $r = n \mod m$} \\
    \ldots
    = \gcd((a^{\gcd(m,n)} - b^{\gcd(m,n)}, 0) = a^{\gcd(m,n)} - b^{\gcd(m,n)}
\end{gather*}

\section*{Question 7}
\subsection*{9).}
If we have $m\text{ $|$ }n$ (meaning that $n$ can be divided by $m$), then n should be divided by all prime factors of n.
\begin{gather*}
    m = {p_{1}}^{a_{1}} \times {p_{2}}^{a_{2}} \ldots \times {p_{k}}^{a_{k}} \\
    m \text{ $|$ } n \Rightarrow {p_{1}}^{a_{1}} \times {p_{2}}^{a_{2}} \ldots \times {p_{k}}^{a_{k}} \text{ $|$ } n \\
    {p_{1}}^{a_{1}} \text{ $|$ } n \wedge {p_{2}}^{a_{2}} \text{ $|$ } n \ldots \wedge {p_{k}}^{a_{k}} \text{ $|$ } n
\end{gather*}

In this problem, we can write number six as its prime factors $2 \times 3$. Therefore, we need need to prove that:
\begin{gather*}
    2 \text{ $|$ } n \times (n+1) \times (n+2) \wedge 3 \text{ $|$ } n \times (n+1) \times (n+2)
\end{gather*}

Since 2 and 3 are coprime, then we can prove each of them individually. Let's start with 2:
\begin{gather*}
    \text{if } n \equiv 0 \mod 2 \to n \times (n+1) \times (n+2) \equiv 0 \mod 2 \\
    \text{if } n \equiv 1 \mod 2 \to n+1 \equiv 2 \equiv 0 \mod 2 \to n \times (n+1) \times (n+2) \equiv 0 \mod 2
\end{gather*}

To say it in simple words, for each real number as $N$, one of the two $N$ or $N+1$ is divided by 2. As for 3:
\begin{gather*}
    \text{if } n \equiv 0 \mod 3 \to n \times (n+1) \times (n+2) \equiv 0 \mod 3 \\
    \text{if } n \equiv 1 \mod 3 \to n+2 \equiv 3 \equiv 0 \mod 3 \to n \times (n+1) \times (n+2) \equiv 0 \mod 3 \\
    \text{if } n \equiv 2 \mod 3 \to n+1 \equiv 3 \equiv 0 \mod 3 \to n \times (n+1) \times (n+2) \equiv 0 \mod 3
\end{gather*}

To say it in simple words, for each real number as $N$, one of the three $N$, $N+1$, or $N+2$ is divided by 3. Now we can say:
\begin{gather*}
    2 \text{ $|$ } n \times (n+1) \times (n+2) \wedge 3 \text{ $|$ } n \times (n+1) \times (n+2) \\
    6 \text{ $|$ } n \times (n+1) \times (n+2)
\end{gather*}

\section*{Question 8}
\subsection*{10).}
Let's say we have $a$ and $b$, first we rewrite them as their prime factors:
\begin{gather*}
    a = {p_{1}}^{e_{1}} {p_{2}}^{e_{2}} \ldots {p_{k}}^{e_{k}} \\
    b = {p_{1}}^{f_{1}} {p_{2}}^{f_{2}} \ldots {p_{k}}^{f_{k}}
\end{gather*}

Now we are going to compute $\gcd(a, b)$ and $\operatorname{lcm}(a, b)$ with these values:
\begin{gather*}
    \gcd(a, b) = {p_{1}}^{\min(e_{1}, f_{1})} {p_{2}}^{\min(e_{2}, f_{2})} \ldots {p_{k}}^{\min(e_{k}, f_{k})} \\
    \operatorname{lcm}(a, b) = {p_{1}}^{\max(e_{1}, f_{1})} {p_{2}}^{\max(e_{2}, f_{2})} \ldots {p_{k}}^{\max(e_{k}, f_{k})} \\
    \gcd(a, b) \times \operatorname{lcm}(a, b) = \prod_{i=1}^{k} {p_{i}}^{\min(e_{i}, f_{i})+\max(e_{i}, f_{i})}
\end{gather*}

Now we know that $\min(a, b) + \max(a, b) = a+b$, therefore we have:
\begin{gather*}
    \gcd(a, b) \times \operatorname{lcm}(a, b) = \prod_{i=1}^{k} {p_{i}}^{e_{i} + f_{i}} \\
    = \prod_{i=1}^{k} {p_{i}}^{e_{i}} {p_{i}}^{f_{i}} \\
    = \prod_{i=1}^{k} {p_{i}}^{e_{i}} \times \prod_{i=1}^{k} {p_{i}}^{f_{i}} = a \times b
\end{gather*}

\end{document}
