\documentclass[12pt]{article}

% packages in use
\usepackage{amsmath, amssymb, graphicx}
\usepackage{array}
\usepackage{geometry}

% global settings
\geometry{a4paper, margin=1in}
\setlength{\parindent}{0pt}

\begin{document}

% title section
\begin{center}
    {\LARGE\textbf{CSE 547: Homework One}} \\[1em]
    {\large Amirhossein Najafizadeh} \\[1em]
    Semester: Fall 2024 \\ 
    SBU ID: 116715544 \\
    Email: Amirhossein.Najafizadeh@stonybrook.edu \\[1em]
    \noindent\rule{\textwidth}{0.6pt}
\end{center}

% answer sheet
\section*{Question 1}
\subsection*{1). 1.10}
First we are going to prove that $Q_{n}$ is equal to $2R_{n-1}+1$. Let's assume that we want to go from $A$ to $B$ by using reverse algorithm. In this case our available moves are:
\begin{gather*}
    B \to A \\
    A \to C \\
    C \to B
\end{gather*}

Our approach is going to be like the Tower of Hanoi's solution:
\begin{gather*}
    \text{move} (n-1) : A \to C \Rightarrow R_{n-1} \\
    \text{move the largest disk} : A \to C \to B \Rightarrow 1 \\
    \text{move} (n-1) : C \to A \Rightarrow R_{n-1} \\
    \text{overall cost is} : R_{n-1} + 1 + R_{n-1} = 2R_{n-1} + 1
\end{gather*}

Now that we proved the first relation, we are going to use the same approach to prove that $R_{n}$ is equal to $Q_{n}+Q_{n-1}+1$. Assume we want to go back to $B$ from $A$:
\begin{gather*}
    \text{move} (n-1) : B \to A \Rightarrow R_{n-1} \\
    \text{move} (n-1) : A \to C \Rightarrow R_{n-1} \\
    \text{move the largest disk} : B \to A \Rightarrow 1 \\
    \text{move n-1 disks (it's like two reverse move in order)} : C \to B \to A \Rightarrow 2R_{n-2}+2 \\
    \text{overall cost is} : R_{n-1} + R_{n-1} + 1 + 2R_{n-2} + 2 \\= (2R_{n-1} + 1) + (2R_{n-2} + 1) + 1 = Q_{n} + Q_{n-1} + 1
\end{gather*}

Not to mention that in all cases when $n=0$, then we don't have any disks to move. Therefore, the equations for $n=0$ remains as $R_{0}=Q_{0}=0$.

\section*{Question 2}
\subsection*{2). 1.14}
As this is like the \textit{Line in the plane} problem, we are going to go with the same approach.
\begin{gather*}
    P(n) = P(n-1) + \text{new spaces created by adding the $n$th plane}
\end{gather*}

Now, for counting the new spaces created by adding the new plane, the nth plane can intersect each of the previous $n-1$ planes in a line.
These intersections divide nth plane into new spaces. So, it is a combination of the new and old planes. Therefore:
\begin{gather*}
    P(n) = P(n-1) + 1 + n + \begin{pmatrix} n \\ 2 \end{pmatrix} \\
    P(n) = P(n-1) + 1 + n + \frac{n \times (n-1)}{2}
\end{gather*}

For solving $P(n)$, we are going to use the \textit{Line in the plane} solution:
\begin{gather*}
    P(n) = P(n-2) + 1 + n + \frac{(n-1) \times (n-2)}{2} + 1 + n + \frac{n \times (n-1)}{2} = \ldots
    P(n) = 1 + n + \frac{n \times (n-1)}{2} + \frac{n \times (n-1) \times (n-2)}{2 \times 3}
\end{gather*}

To test our answer:
\begin{gather*}
    P(1) = 1 \\
    P(2) = 1 + 2 + 1 = 4 \\
    P(3) = 1 + 3 + \frac{3 \times (3-1)}{2} + \frac{3 \times (3-1) \times (3-2)}{6} = 8
\end{gather*}

\section*{Question 3}
\subsection*{3). 1.16}

\section*{Question 4}
\subsection*{4). 1.21}
In order to see if an answer exists for this problem, first we are going to check what is our position after $k$th step. We start from number 1 and each time we move $m$ places to kill the person on that position.
To meet the condition of our problem, we need to make sure that our position is always in $(n, n+1, \ldots, 2n)$th positions. Now let's see what are the available positions in each step:
\begin{table}[h]
    \centering
    \begin{tabular}{|c|c|c|}
        \hline
        step & available positions \\
        \hline
        $1$ & $P(1): m \mod 2n \in (n, \ldots, 2n)$ \\
        \hline
        $2$ & $P(2): P(1) + m \mod {2n-1} \in (n, \ldots, 2n-1)$ \\
        \hline
        $3$ & $P(3): P(2) + m \mod {2n-2} \in (n, \ldots, 2n-2)$ \\
        \hline
        $\ldots$ & $\ldots$ \\
        \hline
        $k$ & $P(k): P(k-1) + m \mod {2n-k+1} \in (n, \ldots, 2n-k+1)$ \\
        \hline
    \end{tabular}
    \caption{available positions after each step}
    \label{tab:sample}
\end{table}

In general:
\begin{gather*}
    P(k) = P(k-1) + m \mod {(2n-k+1)} \\
    = P(k-2) + m \mod {(2n-k)} + m \mod {(2n-k+1)} \\
    = m \mod 2n + m \mod 2n + m \mod {(2n-1)} + \ldots
\end{gather*}

Based on the available positions, $m$ should be a number that $P(k) \text{is always in} [n+1, 2n-k]$. To meet that, it needs a common multiple with each number
in the equation. Therefore, $m$ can be any number that is in the common multiple of number $n+1, n+2, \ldots, 2n$.

\section*{Question 5}
\subsection*{5). 2.14}

\section*{Question 6}
\subsection*{6). 2.21}

\section*{Question 7}
\subsection*{9).}
To solve this problem, we are going to expand our recurrence relation:
\begin{gather*}
    T(n) = T(\frac{N}{2}) + n \to n = 2^k \\
    T(2^k) = T(2^{k-1}) + 2^k = T(2^{k-2}) + 2^{k-1} + 2^k + \ldots \\
    T(2^k) = 1 + 2 + 4 + \cdots + 2^k = \sum_{j=0}^k 2^j
\end{gather*}

For a geometric sequence we have:
\begin{gather*}
    S_{n} = \frac{a_{0}(r^n-1)}{r-1}
\end{gather*}

In this case, for $a_{0}=1$ and ${r=2}$ we have:
\begin{gather*}
    S_{n} = 2^{n+1} - 1
\end{gather*}

Therefore:
\begin{gather*}
    T(2^k) = 2^{k+1} - 1 = 2 \times 2^k - 1 \\
    T(n) = 2n - 1
\end{gather*}


\section*{Question 8}
\subsection*{10).}
In order to prove this equation, we are going to use a \textit{mathematical induction}. In the Fibonacci sequence we have:
\begin{gather*}
    F_{n} = F_{n-1} + F_{n-2} \\
    F_{0}=0, F_{1}=1, F_{2}=1
\end{gather*}

For $n=1$ we have:
\begin{gather*}
    \begin{bmatrix}
        F_{2} & F_{1} \\
        F_{1} & F_{0}
    \end{bmatrix}
    =
    \begin{bmatrix}
        1 & 1 \\
        1 & 0
    \end{bmatrix}
    =
    \left(\begin{bmatrix}
        1 & 1 \\
        1 & 0
    \end{bmatrix}\right)^1
\end{gather*}

Now let's assume for $n-1$ we have:
\begin{gather*}
    \begin{bmatrix}
        F_{n} & F_{n-1} \\
        F_{n-1} & F_{n-2}
    \end{bmatrix}
    =
    \left(\begin{bmatrix}
    1 & 1 \\
    1 & 0
    \end{bmatrix}\right)^{n-1}
\end{gather*}

Now we are going to multiply both sides by \( \begin{bmatrix} 1 & 1 \\ 1 & 0 \end{bmatrix} \):
\begin{gather*}
    \begin{bmatrix}
        F_{n} & F_{n-1} \\
        F_{n-1} & F_{n-2}
    \end{bmatrix}
    \times
    \begin{bmatrix}
        1 & 1 \\
        1 & 1
    \end{bmatrix}
    =
    \begin{bmatrix}
        F_{n}+F_{n-1} & F_{n} \\
        F_{n-1}+F_{n-2} & F_{n-1}
    \end{bmatrix}
    =
    \begin{bmatrix}
        F_{n+1} & F_{n} \\
        F_{n} & F_{n-1}
    \end{bmatrix}
    =
    \left(\begin{bmatrix}
    1 & 1 \\
    1 & 0
    \end{bmatrix}\right)^{n}
\end{gather*}

And now that we proved the $n$th step, we can say by the power of \textit{mathematical induction}:
\begin{gather*}
    \begin{bmatrix}
        F_{n+1} & F_{n} \\
        F_{n} & F_{n-1}
    \end{bmatrix}
    =
    \left(\begin{bmatrix}
    1 & 1 \\
    1 & 0
    \end{bmatrix}\right)^{n}
\end{gather*}

\end{document}
