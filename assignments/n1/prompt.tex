\documentclass[12pt]{article}
\usepackage{amsmath, amssymb, graphicx}
\usepackage{hyperref}
\usepackage{geometry}
\geometry{a4paper, margin=1in}

\begin{document}

\title{CSE 547: Assignment Number 1}
\author{Amirhossein Najafizadeh}
\date{\today}
\maketitle

\section*{Metadata}
Course Name: Discrete Mathematics \\
Semester: Fall 2024 \\
Student SBU Number: 116715544

\section{Example Problems}
\subsection{Problem 1}
\textbf{Statement:} Prove that the sum of the first $n$ natural numbers is given by:
\[
S = \frac{n(n+1)}{2}
\]

\textbf{Solution:} We can prove this using mathematical induction.

\textit{Base Case:} For $n = 1$, the sum $S = 1$, which matches the formula.

\textit{Inductive Step:} Assume the formula is true for $n = k$, i.e.,
\[
S = \frac{k(k+1)}{2}
\]
Now, consider $n = k+1$. The sum for $n = k+1$ is:
\[
S = \frac{k(k+1)}{2} + (k+1) = \frac{k(k+1) + 2(k+1)}{2} = \frac{(k+1)(k+2)}{2}
\]
Thus, the formula holds for $n = k+1$. By induction, the formula is true for all $n \in \mathbb{N}$.`'

\end{document}
